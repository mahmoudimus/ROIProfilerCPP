% Prime Sieve with TeX
% author: jfbu
% date:   2021/07/24
%
% Utilities for
%
% - nestable \xloop macro
% - integer square root
% - replicating tokens a given pre-defined number of times
%
% TeXnicalities:
%
%   ////                                                   \\\\
%   This file only needs base Knuth's tex, i.e. we
%   don't use \numexpr, \dimexpr, etc... for computations.
%   \\\\                                                   ////
%

% activate @ as a letter for being more mysterious
\catcode`@ 11
% and use also _ as a letter to appear even more TeXnical
\catcode`_ 11

% scratch counts usable by any macro for purely temporary things
% they are used by \SetToSqrt
\newcount\tmpcnta
\newcount\tmpcntb
\newcount\tmpcntc

% NESTABLE (and expandable) LOOP
% ==============================

% Plain TeX's \loop has some problems, e.g.
% \def\macro{... do something with \loop...\repeat...}
% then do some \loop...use \macro...\repeat will be broken
% in general as external \loop defines a \body which will be
% overwritten by inner \loop...
% So let's borrow the \xintloop from xintkernel.sty renaming it into xloop
\long\def\xloop #1#2\repeat {#1#2\xloop_again\fi\@gobble {#1#2}}%
\long\def\xloop_again\fi\@gobble #1{\fi #1\xloop_again\fi\@gobble {#1}}%
\long\def\@gobble#1{}%

% INTEGER SQUARE ROOT
% ===================
% input must be non-negative and at most 999999999 
%                  (TeX's max number is 2147483647)
% auxiliary used for not too bad starting point of the algorithm
% count how many tokens but input **must have at most 9 tokens**
% usage: in completely expanding context
\long\def\howmany#1{\expandafter\howmany_i#19876543210\jfbye}%
\long\def\howmany_i#1{\howmany_ii}%
\long\def\howmany_ii#1#2#3#4#5#6#7#8#9{#9\jfbye}%
\long\def\jfbye#1\jfbye{}
%
\def\SetToSqrt#1#2{% #1 = input, #2 a count register for output
    % #1 must be at most 10^9-1
    \tmpcnta#1\relax
    % start with some suitable upper bound
    \tmpcntb
     \ifcase\howmany{\the\tmpcnta}\relax
     %   %       0 digit (never happens)
     \or 3%      1 digit
     \or 10%     2
     \or 32%     3
     \or 100%    4
     \or 317%    5
     \or 1000%   6
     \or 3163%   7
     \or 10000%  8
     \else 31623%9
     \fi\relax
    % and apply *integer* babylonian square root
    % ATTENTION: using Plain TeX's \loop here would make this macro not usable in a \loop...
    \xloop
     \tmpcntc=\tmpcnta
     \divide\tmpcntc\tmpcntb  % TeX's \divide truncates (like Python's // for
                              % positive)
     \advance\tmpcntb\tmpcntc
     \divide\tmpcntb\tw@
    \ifnum\tmpcntb>\tmpcntc
    \repeat
    % do the final assignment
    #2=\tmpcntb
}%

% REPLICATION
% ===========
% 1st arg must be non-negative
% 1st arg must be an explicit number or "\the<count>"
\def\Replicate#1{\romannumeral\expandafter\Rep_#1\endcsname}%
\def\Rep_ #1{\csname Repf#1\Repa}%
\def\Repa #1{\csname Rep#1\Repa}%
\def\Rep\Repa{\endcsname}%
\long\expandafter\def\csname Rep0\endcsname #1%
    {\endcsname{#1#1#1#1#1#1#1#1#1#1}}%
\long\expandafter\def\csname Rep1\endcsname #1%
    {\endcsname{#1#1#1#1#1#1#1#1#1#1}#1}%
\long\expandafter\def\csname Rep2\endcsname #1%
    {\endcsname{#1#1#1#1#1#1#1#1#1#1}#1#1}%
\long\expandafter\def\csname Rep3\endcsname #1%
    {\endcsname{#1#1#1#1#1#1#1#1#1#1}#1#1#1}%
\long\expandafter\def\csname Rep4\endcsname #1%
    {\endcsname{#1#1#1#1#1#1#1#1#1#1}#1#1#1#1}%
\long\expandafter\def\csname Rep5\endcsname #1%
    {\endcsname{#1#1#1#1#1#1#1#1#1#1}#1#1#1#1#1}%
\long\expandafter\def\csname Rep6\endcsname #1%
    {\endcsname{#1#1#1#1#1#1#1#1#1#1}#1#1#1#1#1#1}%
\long\expandafter\def\csname Rep7\endcsname #1%
    {\endcsname{#1#1#1#1#1#1#1#1#1#1}#1#1#1#1#1#1#1}%
\long\expandafter\def\csname Rep8\endcsname #1%
    {\endcsname{#1#1#1#1#1#1#1#1#1#1}#1#1#1#1#1#1#1#1}%
\long\expandafter\def\csname Rep9\endcsname #1%
    {\endcsname{#1#1#1#1#1#1#1#1#1#1}#1#1#1#1#1#1#1#1#1}%
\long\expandafter\def\csname Repf0\endcsname #1{\z@}%
\long\expandafter\def\csname Repf1\endcsname #1{\z@ #1}%
\long\expandafter\def\csname Repf2\endcsname #1{\z@ #1#1}%
\long\expandafter\def\csname Repf3\endcsname #1{\z@ #1#1#1}%
\long\expandafter\def\csname Repf4\endcsname #1{\z@ #1#1#1#1}%
\long\expandafter\def\csname Repf5\endcsname #1{\z@ #1#1#1#1#1}%
\long\expandafter\def\csname Repf6\endcsname #1{\z@ #1#1#1#1#1#1}%
\long\expandafter\def\csname Repf7\endcsname #1{\z@ #1#1#1#1#1#1#1}%
\long\expandafter\def\csname Repf8\endcsname #1{\z@ #1#1#1#1#1#1#1#1}%
\long\expandafter\def\csname Repf9\endcsname #1{\z@ #1#1#1#1#1#1#1#1#1}%
%
% specialized variant optimized for replicating exactly 1000 times
\long\def\ReplicateM#1{\ReplicateC{#1#1#1#1#1#1#1#1#1#1}}%
\long\def\ReplicateC#1{\ReplicateX{#1#1#1#1#1#1#1#1#1#1}}%
\long\def\ReplicateX#1{#1#1#1#1#1#1#1#1#1#1}%
\long\def\ReplicateOneTwoFive#1{%
    #1#1#1#1#1#1#1#1#1#1%
    #1#1#1#1#1#1#1#1#1#1%
    #1#1#1#1#1#1#1#1#1#1%
    #1#1#1#1#1#1#1#1#1#1%
    #1#1#1#1#1#1#1#1#1#1%
%
    #1#1#1#1#1#1#1#1#1#1%
    #1#1#1#1#1#1#1#1#1#1%
    #1#1#1#1#1#1#1#1#1#1%
    #1#1#1#1#1#1#1#1#1#1%
    #1#1#1#1#1#1#1#1#1#1%
%
    #1#1#1#1#1%
    #1#1#1#1#1%
    #1#1#1#1#1%
    #1#1#1#1#1%
    #1#1#1#1#1%
}%
\long\def\ReplicateTwentyOne#1{%
    #1#1#1#1#1#1#1#1#1#1%
    #1#1#1#1#1#1#1#1#1#1%
    #1%
}%
\endinput

